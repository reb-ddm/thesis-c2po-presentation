\usepackage[utf8]{inputenc}

\usepackage{pgfplots}
\usepackage{pgfplotstable}
\usepackage{tikz}
\usepackage{xcolor}
\usetikzlibrary{automata, arrows.meta, shapes.geometric, positioning}

\usepackage{minted}
\usepackage{todonotes}
\usepackage[normalem]{ulem}
\usepackage[export]{adjustbox}



% \newtheorem{theorem}{Theorem}[chapter]
% \newtheorem{corollary}{Corollary}[theorem]
% \newtheorem{lemma}[theorem]{Lemma}
% \newtheorem{proposition}[theorem]{Proposition}
% \theoremstyle{definition}
% \newtheorem{example}[theorem]{Example}
% \newtheorem{definition}[theorem]{Definition}



\newcommand{\ZZ}{\mathbb{Z} \times \mathbb{Z}}
\newcommand{\nequiv}{\not\equiv}
\newcommand{\nequivp}{\nequiv_{\Psi,\T}}
\newcommand{\equivp}{\equiv_{\Psi,\T}}
% \renewcommand{\labelenumii}{\arabic{enumii}.}
\newcommand{\join}{\sqcup}
\newcommand{\meet}{\sqcap}
\newcommand{\widen}{\nabla}
\newcommand{\narrow}{\Delta}
\newcommand{\sem}[1]{\llbracket#1\rrbracket}
\newcommand{\Z}{\mathbb{Z}}
\renewcommand{\L}{\mathcal{L}}
\newcommand{\T}{\mathcal{T}}
\newcommand{\Set}{\mathcal{S}}
\newcommand{\oT}{{\overline{\mathcal{T}}}}
\newcommand{\F}{\mathcal{F}}
\newcommand{\A}{\mathcal{A}}
\newcommand{\X}{\mathcal{X}}
\newcommand{\V}{\mathcal{V}}
\newcommand{\malloc}{\textsf{malloc}}
\newcommand{\nf}{\textsf{nf}}
\newcommand{\restr}[2]{\left.\kern-\nulldelimiterspace#1\vphantom{|}\right|_{#2}}
\newcommand{\otau}{\overline{\tau}}
\newcommand{\angl}[1]{\langle#1\rangle}
% \newcommand{\widen}{\mathop{{\sqcup}\hspace*{-0.6em}\raisebox{.4ex}{\setlength{\unitlength}{1em}\line(1,0){.53}}}}
% \newcommand{\narrow}{\mathop{{\sqcap}\hspace*{-0.6em}\raisebox{.9ex}{\setlength{\unitlength}{1em}\line(1,0){.50}}}}
\newcommand{\ignore}[1]{}
\newcommand{\cpo}{\textsf{C-2PO}}
\newcommand{\goblint}{\textsc{Goblint}}
\newcommand{\find}[1]{\textsf{find}(#1)}
\newcommand{\union}[3]{\textsf{union}(#1, #2, #3)}
\newcommand{\closure}[4]{\textsf{closure}(#4, #1 = #2 + #3)}
\newcommand{\ext}[2]{\textsf{ext}\,{#1}\,{#2}}
\newcommand{\enter}{\textsf{enter}}
\newcommand{\combine}{\textsf{combine}}
\newcommand{\base}{\emph{base}}
\newcommand{\vareq}{\emph{var\_eq}}
\newcommand{\cpou}{$c$-$2po_1$}
\newcommand{\cpod}{$c$-$2po_2$}
\newcommand{\cpot}{$c$-$2po_3$}
\newcommand{\cpoq}{$c$-$2po_4$}


% Define TUM corporate design colors
% Taken from http://portal.mytum.de/corporatedesign/index_print/vorlagen/index_farben
\definecolor{TUMBlue}{HTML}{0065BD}
\definecolor{TUMSecondaryBlue}{HTML}{005293}
\definecolor{TUMSecondaryBlue2}{HTML}{003359}
\definecolor{TUMBlack}{HTML}{000000}
\definecolor{TUMWhite}{HTML}{FFFFFF}
\definecolor{TUMDarkGray}{HTML}{333333}
\definecolor{TUMGray}{HTML}{808080}
\definecolor{TUMLightGray}{HTML}{CCCCC6}
\definecolor{TUMAccentGray}{HTML}{DAD7CB}
\definecolor{TUMAccentOrange}{HTML}{E37222}
\definecolor{TUMAccentGreen}{HTML}{A2AD00}
\definecolor{TUMAccentLightBlue}{HTML}{98C6EA}
\definecolor{TUMAccentBlue}{HTML}{64A0C8}
